\documentclass[12pt]{article}


\topmargin 0pt
\advance \topmargin by -\headheight
\advance \topmargin by -\headsep
\textheight 8.9in
\oddsidemargin 0pt
\evensidemargin \oddsidemargin
\marginparwidth 0.5in
\textwidth 6.5in

\parindent 0in
\parskip 1.5ex
%\renewcommand{\baselinestretch}{1.25}

\usepackage{xr-hyper}
\def\renewtheorem{}

% Note: this has been tested using MiKTeX 2.9. If you are getting errors, update your packages.
% LTeX: enabled=false

%%% Packages %%%
%\usepackage{setspace} % Double spaces document. Footnotes,
% figures, and tables will still be single spaced, however.
%\doublespacing
%\singlespacing
%\onehalfspacing
% \setstretch{1.5} % set double spacing to 1.5 or anything else.


\usepackage[T1]{fontenc}
\usepackage[utf8]{inputenc}
\usepackage{mathtools}

\providecommand{\hmmax}{0}
\providecommand{\bmmax}{0}
\usepackage{amssymb,mathrsfs}% Typical maths resource packages
\usepackage{amsthm}
\usepackage{bm}
\usepackage{scalerel}
\usepackage{nicefrac}
\usepackage{microtype} 
\usepackage[shortlabels]{enumitem}
\usepackage{graphicx}
\usepackage{epstopdf}
\DeclareGraphicsExtensions{.eps,.png,.jpg,.pdf}

\usepackage{url}
\usepackage{colortbl}
\usepackage{booktabs}
\usepackage{multirow}
\usepackage{colortbl,xcolor}
\usepackage[normalem]{ulem}
\usepackage{xparse,xstring}
\usepackage{calc}
\usepackage{etoolbox}

\makeatletter
\@ifpackageloaded{natbib}{
	\relax
}{
	\usepackage{cite}
}
\makeatother

%\usepackage{pstricks}
%\usepackage{psfrag}
%\usepackage{syntonly}
%\syntaxonly
%\usepackage[style=base]{caption}
%\captionsetup{
%format = plain,
%font = footnotesize,
%labelfont = sc
%}


\usepackage{array}
\newcolumntype{L}[1]{>{\raggedright\let\newline\\\arraybackslash\hspace{0pt}}m{#1}}
\newcolumntype{C}[1]{>{\centering\let\newline\\\arraybackslash\hspace{0pt}}m{#1}}
\newcolumntype{R}[1]{>{\raggedleft\let\newline\\\arraybackslash\hspace{0pt}}m{#1}}

\makeatletter
\let\MYcaption\@makecaption
\makeatother
\usepackage[font=footnotesize]{subcaption}
\makeatletter
\let\@makecaption\MYcaption
\makeatother

\usepackage{glossaries}
\makeatletter
\sfcode`\.1006

% copy old \gls and \glspl '
\let\oldgls\gls
\let\oldglspl\glspl

% define a non space skipping version of \@ifnextchar
\newcommand\fussy@ifnextchar[3]{%
	\let\reserved@d=#1%
	\def\reserved@a{#2}%
	\def\reserved@b{#3}%
	\futurelet\@let@token\fussy@ifnch}
\def\fussy@ifnch{%
	\ifx\@let@token\reserved@d
		\let\reserved@c\reserved@a
	\else
		\let\reserved@c\reserved@b
	\fi
	\reserved@c}

\renewcommand{\gls}[1]{%
\oldgls{#1}\fussy@ifnextchar.{\@checkperiod}{\@}}
\renewcommand{\glspl}[1]{%
\oldglspl{#1}\fussy@ifnextchar.{\@checkperiod}{\@}}

\newcommand{\@checkperiod}[1]{%
	\ifnum\sfcode`\.=\spacefactor\else#1\fi
}

% '
\robustify{\gls}
\robustify{\glspl}
\makeatother

\newacronym{wrt}{w.r.t.}{with respect to}
\newacronym{RHS}{R.H.S.}{right-hand side}
\newacronym{LHS}{L.H.S.}{left-hand side}
\newacronym{iid}{i.i.d.}{independent and identically distributed}
\newacronym{SOTA}{SOTA}{state-of-the-art}
%\newacronym{MIMO}{MIMO}{mulitple-input multiple-output}
%\newacronym{AOA}{AOA}{angle-of-arrival}
%\newacronym{AOD}{AOD}{angle-of-departure}
%\newacronym{LOS}{LOS}{line-of-sight}
%\newacronym{NLOS}{NLOS}{non-line-of-sight}
%\newacronym{TOA}{TOA}{time-of-arrival}
%\newacronym{TDOA}{TDOA}{time-difference-of-arrival}
%\newacronym{RSS}{RSS}{received signal strength}
%\newacronym{GNSS}{GNSS}{Global Navigation Satellite System}
%\newacronym{GSP}{GSP}{graph signal processing}
%\newacronym{ML}{ML}{machine learning}


%put the float package before hyperref and algorithm package after hyperref for hyperref to work correctly with algorithm
\usepackage{float}

\ifx\notloadhyperref\undefined
	\ifx\loadbibentry\undefined
		\usepackage[hidelinks,hypertexnames=false]{hyperref}
	\else
		\usepackage{bibentry}
		\makeatletter\let\saved@bibitem\@bibitem\makeatother
		\usepackage[hidelinks,hypertexnames=false]{hyperref}
		\makeatletter\let\@bibitem\saved@bibitem\makeatother
	\fi
\else
	\ifx\loadbibentry\undefined
		\relax
	\else
		\usepackage{bibentry}
	\fi
\fi

\usepackage[capitalize]{cleveref}
\crefname{equation}{}{}
\Crefname{equation}{}{}
\crefname{claim}{claim}{claims}
\crefname{step}{step}{steps}
\crefname{line}{line}{lines}
\crefname{condition}{condition}{conditions}
\crefname{dmath}{}{}
\crefname{dseries}{}{}
\crefname{dgroup}{}{}

\crefname{Problem}{Problem}{Problems}
\crefformat{Problem}{Problem~#2#1#3}
\crefrangeformat{Problem}{Problems~#3#1#4 to~#5#2#6}

\crefname{Theorem}{Theorem}{Theorems}
\crefname{Corollary}{Corollary}{Corollaries}
\crefname{Proposition}{Proposition}{Propositions}
\crefname{Lemma}{Lemma}{Lemmas}
\crefname{Definition}{Definition}{Definitions}
\crefname{Example}{Example}{Examples}
\crefname{Assumption}{Assumption}{Assumptions}
\crefname{Remark}{Remark}{Remarks}
\crefname{Rem}{Remark}{Remarks}
\crefname{remarks}{Remarks}{Remarks}
\crefname{Appendix}{Appendix}{Appendices}
\crefname{Supplement}{Supplement}{Supplements}
\crefname{Exercise}{Exercise}{Exercises}
\crefname{Theorem_A}{Theorem}{Theorems}
\crefname{Corollary_A}{Corollary}{Corollaries}
\crefname{Proposition_A}{Proposition}{Propositions}
\crefname{Lemma_A}{Lemma}{Lemmas}
\crefname{Definition_A}{Definition}{Definitions}

\usepackage{crossreftools}
\ifx\notloadhyperref\undefined
	\pdfstringdefDisableCommands{%
		\let\Cref\crtCref
		\let\cref\crtcref
	}
\else
	\relax
\fi

\usepackage{algorithm}
\usepackage{algpseudocode}

%may cause conflict with some packages like tikz, include manually if desired
%load after hyperref
\ifx\loadbreqn\undefined
	\relax
\else
	\usepackage{breqn}
\fi



%%%%%%%%%%%%%%%%%%%%%%%%%%%%%%%%%%%%%%%%%%%%%%%%


\interdisplaylinepenalty=2500   % To restore IEEEtran ability to automatically break
% within multiline equations, when using amsmath.

%%%%%%%%%%%%%%%%%%%%%%%%%%%%%%%%%%%%%%%%

%Theorem declarations
% \def\renewtheorem in your document if you want to redefine these theorem environments.
% Alternatively, call \clearthms

\makeatletter
\def\cleartheorem#1{%
    \expandafter\let\csname#1\endcsname\relax
    \expandafter\let\csname c@#1\endcsname\relax
}
\def\clearthms#1{ \@for\tname:=#1\do{\cleartheorem\tname} }
\makeatother

\ifx\renewtheorem\undefined
	% for use in main body
	\ifx\useTheoremCounter\undefined
		\newtheorem{Theorem}{Theorem}
		\newtheorem{Corollary}{Corollary}
		\newtheorem{Proposition}{Proposition}
		\newtheorem{Lemma}{Lemma}
	\else
		\newtheorem{Theorem}{Theorem}
		\newtheorem{Corollary}[Theorem]{Corollary}
		\newtheorem{Proposition}[Theorem]{Proposition}
	\fi

	\newtheorem{Definition}{Definition}
	\newtheorem{Example}{Example}
	\newtheorem{Remark}{Remark}
	\newtheorem{Assumption}{Assumption}
	\newtheorem{Exercise}{Exercise}
	\newtheorem{Problem}{Problem}

	% for use in the appendix
	\newtheorem{Theorem_A}{Theorem}[section]
	\newtheorem{Corollary_A}{Corollary}[section]
	\newtheorem{Proposition_A}{Proposition}[section]
	\newtheorem{Lemma_A}{Lemma}[section]
	\newtheorem{Definition_A}{Definition}[section]
	\newtheorem{Example_A}{Example}[section]
	\newtheorem{Remark_A}{Remark}[section]
	\newtheorem{Assumption_A}{Assumption}[section]
	\newtheorem{Exercise_A}{Exercise}[section]
\fi

% Remarks
\theoremstyle{remark}
\newtheorem{Rem}{Remark}
\theoremstyle{plain}

\newenvironment{remarks}{
	\begin{list}{\textit{Remark} \arabic{Rem}:~}{
			\setcounter{enumi}{\value{Rem}}
			\usecounter{Rem}
			\setcounter{Rem}{\value{enumi}}
			\setlength\labelwidth{0in}
			\setlength\labelsep{0in}
			\setlength\leftmargin{0in}
			\setlength\listparindent{0in}
			\setlength\itemindent{15pt}
		}
		}{
	\end{list}
}


% Special Headings
%\newtheorem*{Prop1}{Proposition 1} %needs amsthm

%\newtheoremstyle{nonum}{}{}{\itshape}{}{\bfseries}{.}{ }{#1 (\mdseries #3)}
%\theoremstyle{nonum}
%\newtheorem{Example**}{Example 1}

\newcommand{\EndExample}{{$\square$}}
%\renewcommand{\QED}{\QEDopen} % changes end of proof box to open box.

\newcommand{\qednew}{\nobreak \ifvmode \relax \else
		\ifdim\lastskip<1.5em \hskip-\lastskip
			\hskip1.5em plus0em minus0.5em \fi \nobreak
		\vrule height0.75em width0.5em depth0.25em\fi}


% achieves the functionality of \tag for subequations environment
\makeatletter
\newenvironment{varsubequations}[1]
{%
	\addtocounter{equation}{-1}%
	\begin{subequations}
		\renewcommand{\theparentequation}{#1}%
		\def\@currentlabel{#1}%
		}
		{%
	\end{subequations}\ignorespacesafterend
}
\makeatother


\newcommand{\ml}[1]{\begin{multlined}[t]#1\end{multlined}}
\newcommand{\nn}{\nonumber\\ }

% Move down subscripts for some symbols like \chi
\NewDocumentCommand{\movedownsub}{e{^_}}{%
	\IfNoValueTF{#1}{%
		\IfNoValueF{#2}{^{}}% neither ^ nor _, do nothing; if no ^ but _, add ^{}
	}{%
		^{#1}% add superscript if present
	}%
	\IfNoValueF{#2}{_{#2}}% add subscript if present
}

% chi
\let\latexchi\chi
\RenewDocumentCommand{\chi}{}{\latexchi\movedownsub}


%Number sets
\newcommand{\Real}{\mathbb{R}}
\newcommand{\Nat}{\mathbb{N}}
\newcommand{\Rat}{\mathbb{Q}}
\newcommand{\Complex}{\mathbb{C}}

% imaginary number i
\newcommand{\iu}{\mathfrak{i}\mkern1mu}


% Calligraphic stuff
\newcommand{\calA}{\mathcal{A}}
\newcommand{\calB}{\mathcal{B}}
\newcommand{\calC}{\mathcal{C}}
\newcommand{\calD}{\mathcal{D}}
\newcommand{\calE}{\mathcal{E}}
\newcommand{\calF}{\mathcal{F}}
\newcommand{\calG}{\mathcal{G}}
\newcommand{\calH}{\mathcal{H}}
\newcommand{\calI}{\mathcal{I}}
\newcommand{\calJ}{\mathcal{J}}
\newcommand{\calK}{\mathcal{K}}
\newcommand{\calL}{\mathcal{L}}
\newcommand{\calM}{\mathcal{M}}
\newcommand{\calN}{\mathcal{N}}
\newcommand{\calO}{\mathcal{O}}
\newcommand{\calP}{\mathcal{P}}
\newcommand{\calQ}{\mathcal{Q}}
\newcommand{\calR}{\mathcal{R}}
\newcommand{\calS}{\mathcal{S}}
\newcommand{\calT}{\mathcal{T}}
\newcommand{\calU}{\mathcal{U}}
\newcommand{\calV}{\mathcal{V}}
\newcommand{\calW}{\mathcal{W}}
\newcommand{\calX}{\mathcal{X}}
\newcommand{\calY}{\mathcal{Y}}
\newcommand{\calZ}{\mathcal{Z}}

% Boldface stuff
\newcommand{\ba}{\mathbf{a}}
\newcommand{\bA}{\mathbf{A}}
\newcommand{\bb}{\mathbf{b}}
\newcommand{\bB}{\mathbf{B}}
\newcommand{\bc}{\mathbf{c}}
\newcommand{\bC}{\mathbf{C}}
\newcommand{\bd}{\mathbf{d}}
\newcommand{\bD}{\mathbf{D}}
\newcommand{\be}{\mathbf{e}}
\newcommand{\bE}{\mathbf{E}}
\newcommand{\boldf}{\mathbf{f}}
\newcommand{\bF}{\mathbf{F}}
\newcommand{\bg}{\mathbf{g}}
\newcommand{\bG}{\mathbf{G}}
\newcommand{\bh}{\mathbf{h}}
\newcommand{\bH}{\mathbf{H}}
\newcommand{\bi}{\mathbf{i}}
\newcommand{\bI}{\mathbf{I}}
\newcommand{\bj}{\mathbf{j}}
\newcommand{\bJ}{\mathbf{J}}
\newcommand{\bk}{\mathbf{k}}
\newcommand{\bK}{\mathbf{K}}
\newcommand{\bl}{\mathbf{l}}
\newcommand{\bL}{\mathbf{L}}
\newcommand{\boldm}{\mathbf{m}}
\newcommand{\bM}{\mathbf{M}}
\newcommand{\bn}{\mathbf{n}}
\newcommand{\bN}{\mathbf{N}}
\newcommand{\bo}{\mathbf{o}}
\newcommand{\bO}{\mathbf{O}}
\newcommand{\bp}{\mathbf{p}}
\newcommand{\bP}{\mathbf{P}}
\newcommand{\bq}{\mathbf{q}}
\newcommand{\bQ}{\mathbf{Q}}
\newcommand{\br}{\mathbf{r}}
\newcommand{\bR}{\mathbf{R}}
\newcommand{\bs}{\mathbf{s}}
\newcommand{\bS}{\mathbf{S}}
\newcommand{\bt}{\mathbf{t}}
\newcommand{\bT}{\mathbf{T}}
\newcommand{\bu}{\mathbf{u}}
\newcommand{\bU}{\mathbf{U}}
\newcommand{\bv}{\mathbf{v}}
\newcommand{\bV}{\mathbf{V}}
\newcommand{\bw}{\mathbf{w}}
\newcommand{\bW}{\mathbf{W}}
\newcommand{\bx}{\mathbf{x}}
\newcommand{\bX}{\mathbf{X}}
\newcommand{\by}{\mathbf{y}}
\newcommand{\bY}{\mathbf{Y}}
\newcommand{\bz}{\mathbf{z}}
\newcommand{\bZ}{\mathbf{Z}}


\newcommand{\mba}{\bm{a}}
\newcommand{\mbA}{\bm{A}}
\newcommand{\mbb}{\bm{b}}
\newcommand{\mbB}{\bm{B}}
\newcommand{\mbc}{\bm{c}}
\newcommand{\mbC}{\bm{C}}
\newcommand{\mbd}{\bm{d}}
\newcommand{\mbD}{\bm{D}}
\newcommand{\mbe}{\bm{e}}
\newcommand{\mbE}{\bm{E}}
\newcommand{\mbf}{\bm{f}}
\newcommand{\mbF}{\bm{F}}
\newcommand{\mbg}{\bm{g}}
\newcommand{\mbG}{\bm{G}}
\newcommand{\mbh}{\bm{h}}
\newcommand{\mbH}{\bm{H}}
\newcommand{\mbi}{\bm{i}}
\newcommand{\mbI}{\bm{I}}
\newcommand{\mbj}{\bm{j}}
\newcommand{\mbJ}{\bm{J}}
\newcommand{\mbk}{\bm{k}}
\newcommand{\mbK}{\bm{K}}
\newcommand{\mbl}{\bm{l}}
\newcommand{\mbL}{\bm{L}}
\newcommand{\mbm}{\bm{m}}
\newcommand{\mbM}{\bm{M}}
\newcommand{\mbn}{\bm{n}}
\newcommand{\mbN}{\bm{N}}
\newcommand{\mbo}{\bm{o}}
\newcommand{\mbO}{\bm{O}}
\newcommand{\mbp}{\bm{p}}
\newcommand{\mbP}{\bm{P}}
\newcommand{\mbq}{\bm{q}}
\newcommand{\mbQ}{\bm{Q}}
\newcommand{\mbr}{\bm{r}}
\newcommand{\mbR}{\bm{R}}
\newcommand{\mbs}{\bm{s}}
\newcommand{\mbS}{\bm{S}}
\newcommand{\mbt}{\bm{t}}
\newcommand{\mbT}{\bm{T}}
\newcommand{\mbu}{\bm{u}}
\newcommand{\mbU}{\bm{U}}
\newcommand{\mbv}{\bm{v}}
\newcommand{\mbV}{\bm{V}}
\newcommand{\mbw}{\bm{w}}
\newcommand{\mbW}{\bm{W}}
\newcommand{\mbx}{\bm{x}}
\newcommand{\mbX}{\bm{X}}
\newcommand{\mby}{\bm{y}}
\newcommand{\mbY}{\bm{Y}}
\newcommand{\mbz}{\bm{z}}
\newcommand{\mbZ}{\bm{Z}}

% Numbers bb font
\newcommand{\bbA}{\mathbb{A}}
\newcommand{\bbB}{\mathbb{B}}
\newcommand{\bbC}{\mathbb{C}}
\newcommand{\bbD}{\mathbb{D}}
\newcommand{\bbE}{\mathbb{E}}
\newcommand{\bbF}{\mathbb{F}}
\newcommand{\bbG}{\mathbb{G}}
\newcommand{\bbH}{\mathbb{H}}
\newcommand{\bbI}{\mathbb{I}}
\newcommand{\bbJ}{\mathbb{J}}
\newcommand{\bbK}{\mathbb{K}}
\newcommand{\bbL}{\mathbb{L}}
\newcommand{\bbM}{\mathbb{M}}
\newcommand{\bbN}{\mathbb{N}}
\newcommand{\bbO}{\mathbb{O}}
\newcommand{\bbP}{\mathbb{P}}
\newcommand{\bbQ}{\mathbb{Q}}
\newcommand{\bbR}{\mathbb{R}}
\newcommand{\bbS}{\mathbb{S}}
\newcommand{\bbT}{\mathbb{T}}
\newcommand{\bbU}{\mathbb{U}}
\newcommand{\bbV}{\mathbb{V}}
\newcommand{\bbW}{\mathbb{W}}
\newcommand{\bbX}{\mathbb{X}}
\newcommand{\bbY}{\mathbb{Y}}
\newcommand{\bbZ}{\mathbb{Z}}

% Mathfrak font
\newcommand{\frakA}{\mathfrak{A}}
\newcommand{\frakB}{\mathfrak{B}}
\newcommand{\frakC}{\mathfrak{C}}
\newcommand{\frakD}{\mathfrak{D}}
\newcommand{\frakE}{\mathfrak{E}}
\newcommand{\frakF}{\mathfrak{F}}
\newcommand{\frakG}{\mathfrak{G}}
\newcommand{\frakH}{\mathfrak{H}}
\newcommand{\frakI}{\mathfrak{I}}
\newcommand{\frakJ}{\mathfrak{J}}
\newcommand{\frakK}{\mathfrak{K}}
\newcommand{\frakL}{\mathfrak{L}}
\newcommand{\frakM}{\mathfrak{M}}
\newcommand{\frakN}{\mathfrak{N}}
\newcommand{\frakO}{\mathfrak{O}}
\newcommand{\frakP}{\mathfrak{P}}
\newcommand{\frakQ}{\mathfrak{Q}}
\newcommand{\frakR}{\mathfrak{R}}
\newcommand{\frakS}{\mathfrak{S}}
\newcommand{\frakT}{\mathfrak{T}}
\newcommand{\frakU}{\mathfrak{U}}
\newcommand{\frakV}{\mathfrak{V}}
\newcommand{\frakW}{\mathfrak{W}}
\newcommand{\frakX}{\mathfrak{X}}
\newcommand{\frakY}{\mathfrak{Y}}
\newcommand{\frakZ}{\mathfrak{Z}}

% Mathscr
\newcommand{\scA}{\mathscr{A}}
\newcommand{\scB}{\mathscr{B}}
\newcommand{\scC}{\mathscr{C}}
\newcommand{\scD}{\mathscr{D}}
\newcommand{\scE}{\mathscr{E}}
\newcommand{\scF}{\mathscr{F}}
\newcommand{\scG}{\mathscr{G}}
\newcommand{\scH}{\mathscr{H}}
\newcommand{\scI}{\mathscr{I}}
\newcommand{\scJ}{\mathscr{J}}
\newcommand{\scK}{\mathscr{K}}
\newcommand{\scL}{\mathscr{L}}
\newcommand{\scM}{\mathscr{M}}
\newcommand{\scN}{\mathscr{N}}
\newcommand{\scO}{\mathscr{O}}
\newcommand{\scP}{\mathscr{P}}
\newcommand{\scQ}{\mathscr{Q}}
\newcommand{\scR}{\mathscr{R}}
\newcommand{\scS}{\mathscr{S}}
\newcommand{\scT}{\mathscr{T}}
\newcommand{\scU}{\mathscr{U}}
\newcommand{\scV}{\mathscr{V}}
\newcommand{\scW}{\mathscr{W}}
\newcommand{\scX}{\mathscr{X}}
\newcommand{\scY}{\mathscr{Y}}
\newcommand{\scZ}{\mathscr{Z}}


% define some useful uppercase Greek letters in regular and bold sf
\DeclareSymbolFont{bsfletters}{OT1}{cmss}{bx}{n}
\DeclareSymbolFont{ssfletters}{OT1}{cmss}{m}{n}
\DeclareMathSymbol{\bsfGamma}{0}{bsfletters}{'000}
\DeclareMathSymbol{\ssfGamma}{0}{ssfletters}{'000}
\DeclareMathSymbol{\bsfDelta}{0}{bsfletters}{'001}
\DeclareMathSymbol{\ssfDelta}{0}{ssfletters}{'001}
\DeclareMathSymbol{\bsfTheta}{0}{bsfletters}{'002}
\DeclareMathSymbol{\ssfTheta}{0}{ssfletters}{'002}
\DeclareMathSymbol{\bsfLambda}{0}{bsfletters}{'003}
\DeclareMathSymbol{\ssfLambda}{0}{ssfletters}{'003}
\DeclareMathSymbol{\bsfXi}{0}{bsfletters}{'004}
\DeclareMathSymbol{\ssfXi}{0}{ssfletters}{'004}
\DeclareMathSymbol{\bsfPi}{0}{bsfletters}{'005}
\DeclareMathSymbol{\ssfPi}{0}{ssfletters}{'005}
\DeclareMathSymbol{\bsfSigma}{0}{bsfletters}{'006}
\DeclareMathSymbol{\ssfSigma}{0}{ssfletters}{'006}
\DeclareMathSymbol{\bsfUpsilon}{0}{bsfletters}{'007}
\DeclareMathSymbol{\ssfUpsilon}{0}{ssfletters}{'007}
\DeclareMathSymbol{\bsfPhi}{0}{bsfletters}{'010}
\DeclareMathSymbol{\ssfPhi}{0}{ssfletters}{'010}
\DeclareMathSymbol{\bsfPsi}{0}{bsfletters}{'011}
\DeclareMathSymbol{\ssfPsi}{0}{ssfletters}{'011}
\DeclareMathSymbol{\bsfOmega}{0}{bsfletters}{'012}
\DeclareMathSymbol{\ssfOmega}{0}{ssfletters}{'012}


% Greek
\newcommand{\balpha}{\bm{\alpha}}
\newcommand{\bbeta}{\bm{\beta}}
\newcommand{\bgamma}{\bm{\gamma}}
\newcommand{\bdelta}{\bm{\delta}}
\newcommand{\btheta}{\bm{\theta}}
\newcommand{\bmu}{\bm{\mu}}
\newcommand{\bnu}{\bm{\nu}}
\newcommand{\btau}{\bm{\tau}}
\newcommand{\bpi}{\bm{\pi}}
\newcommand{\bepsilon}{\bm{\epsilon}}
\newcommand{\bvarepsilon}{\bm{\varepsilon}}
\newcommand{\bsigma}{\bm{\sigma}}
\newcommand{\bvarsigma}{\bm{\varsigma}}
\newcommand{\bzeta}{\bm{\zeta}}
\newcommand{\bmeta}{\bm{\eta}}
\newcommand{\bkappa}{\bm{\kappa}}
\newcommand{\bchi}{\bm{\latexchi}\movedownsub}
\newcommand{\bphi}{\bm{\phi}}
\newcommand{\bpsi}{\bm{\psi}}
\newcommand{\bomega}{\bm{\omega}}
\newcommand{\bxi}{\bm{\xi}}
\newcommand{\blambda}{\bm{\lambda}}
\newcommand{\brho}{\bm{\rho}}

\newcommand{\bGamma}{\bm{\Gamma}}
\newcommand{\bLambda}{\bm{\Lambda}}
\newcommand{\bSigma	}{\bm{\Sigma}}
\newcommand{\bPsi}{\bm{\Psi}}
\newcommand{\bDelta}{\bm{\Delta}}
\newcommand{\bXi}{\bm{\Xi}}
\newcommand{\bUpsilon}{\bm{\Upsilon}}
\newcommand{\bOmega}{\bm{\Omega}}
\newcommand{\bPhi}{\bm{\Phi}}
\newcommand{\bPi}{\bm{\Pi}}
\newcommand{\bTheta}{\bm{\Theta}}

\newcommand{\talpha}{\widetilde{\alpha}}
\newcommand{\tbeta}{\widetilde{\beta}}
\newcommand{\tgamma}{\widetilde{\gamma}}
\newcommand{\tdelta}{\widetilde{\delta}}
\newcommand{\ttheta}{\widetilde{\theta}}
\newcommand{\tmu}{\widetilde{\mu}}
\newcommand{\tnu}{\widetilde{\nu}}
\newcommand{\ttau}{\widetilde{\tau}}
\newcommand{\tpi}{\widetilde{\pi}}
\newcommand{\tepsilon}{\widetilde{\epsilon}}
\newcommand{\tvarepsilon}{\widetilde{\varepsilon}}
\newcommand{\tsigma}{\widetilde{\sigma}}
\newcommand{\tvarsigma}{\widetilde{\varsigma}}
\newcommand{\tzeta}{\widetilde{\zeta}}
\newcommand{\tmeta}{\widetilde{\eta}}
\newcommand{\tkappa}{\widetilde{\kappa}}
\newcommand{\tchi}{\widetilde{\latexchi}\movedownsub}
\newcommand{\tphi}{\widetilde{\phi}}
\newcommand{\tpsi}{\widetilde{\psi}}
\newcommand{\tomega}{\widetilde{\omega}}
\newcommand{\txi}{\widetilde{\xi}}
\newcommand{\tlambda}{\widetilde{\lambda}}
\newcommand{\trho}{\widetilde{\rho}}

\newcommand{\tGamma}{\widetilde{\Gamma}}
\newcommand{\tDelta}{\widetilde{\Delta}}
\newcommand{\tTheta}{\widetilde{\Theta}}
\newcommand{\tPi}{\widetilde{\Pi}}
\newcommand{\tSigma}{\widetilde{\Sigma}}
\newcommand{\tPhi}{\widetilde{\Phi}}
\newcommand{\tPsi}{\widetilde{\Psi}}
\newcommand{\tOmega}{\widetilde{\Omega}}
\newcommand{\tXi}{\widetilde{\Xi}}
\newcommand{\tLambda}{\widetilde{\Lambda}}

\newcommand{\tbalpha}{\widetilde{\balpha}}
\newcommand{\tbbeta}{\widetilde{\bbeta}}
\newcommand{\tbgamma}{\widetilde{\bgamma}}
\newcommand{\tbdelta}{\widetilde{\bdelta}}
\newcommand{\tbtheta}{\widetilde{\btheta}}
\newcommand{\tbmu}{\widetilde{\bmu}}
\newcommand{\tbnu}{\widetilde{\bnu}}
\newcommand{\tbtau}{\widetilde{\btbau}}
\newcommand{\tbpi}{\widetilde{\bpi}}
\newcommand{\tbepsilon}{\widetilde{\bepsilon}}
\newcommand{\tbvarepsilon}{\widetilde{\bvarepsilon}}
\newcommand{\tbsigma}{\widetilde{\bsigma}}
\newcommand{\tbvarsigma}{\widetilde{\bvarsigma}}
\newcommand{\tbzeta}{\widetilde{\bzeta}}
\newcommand{\tbmeta}{\widetilde{\beta}}
\newcommand{\tbkappa}{\widetilde{\bkappa}}
\newcommand{\tbchi}{\widetilde\bm{\latexchi}\movedownsub}
\newcommand{\tbphi}{\widetilde{\bphi}}
\newcommand{\tbpsi}{\widetilde{\bpsi}}
\newcommand{\tbomega}{\widetilde{\bomega}}
\newcommand{\tbxi}{\widetilde{\bxi}}
\newcommand{\tblambda}{\widetilde{\blambda}}
\newcommand{\tbrho}{\widetilde{\brho}}

\newcommand{\tbGamma}{\widetilde{\bGamma}}
\newcommand{\tbDelta}{\widetilde{\bDelta}}
\newcommand{\tbTheta}{\widetilde{\bTheta}}
\newcommand{\tbPi}{\widetilde{\bPi}}
\newcommand{\tbSigma}{\widetilde{\bSigma}}
\newcommand{\tbPhi}{\widetilde{\bPhi}}
\newcommand{\tbPsi}{\widetilde{\bPsi}}
\newcommand{\tbOmega}{\widetilde{\bOmega}}
\newcommand{\tbXi}{\widetilde{\bXi}}
\newcommand{\tbLambda}{\widetilde{\bLambda}}

\newcommand{\halpha}{\widehat{\alpha}}
\newcommand{\hbeta}{\widehat{\beta}}
\newcommand{\hgamma}{\widehat{\gamma}}
\newcommand{\hdelta}{\widehat{\delta}}
\newcommand{\htheta}{\widehat{\theta}}
\newcommand{\hmu}{\widehat{\mu}}
\newcommand{\hnu}{\widehat{\nu}}
\newcommand{\htau}{\widehat{\tau}}
\newcommand{\hpi}{\widehat{\pi}}
\newcommand{\hepsilon}{\widehat{\epsilon}}
\newcommand{\hvarepsilon}{\widehat{\varepsilon}}
\newcommand{\hsigma}{\widehat{\sigma}}
\newcommand{\hvarsigma}{\widehat{\varsigma}}
\newcommand{\hzeta}{\widehat{\zeta}}
\newcommand{\heta}{\widehat{\eta}}
\newcommand{\hkappa}{\widehat{\kappa}}
\newcommand{\hchi}{\widehat{\latexchi}\movedownsub}
\newcommand{\hphi}{\widehat{\phi}}
\newcommand{\hpsi}{\widehat{\psi}}
\newcommand{\homega}{\widehat{\omega}}
\newcommand{\hxi}{\widehat{\xi}}
\newcommand{\hlambda}{\widehat{\lambda}}
\newcommand{\hrho}{\widehat{\rho}}

\newcommand{\hGamma}{\widehat{\Gamma}}
\newcommand{\hDelta}{\widehat{\Delta}}
\newcommand{\hTheta}{\widehat{\Theta}}
\newcommand{\hPi}{\widehat{\Pi}}
\newcommand{\hSigma}{\widehat{\Sigma}}
\newcommand{\hPhi}{\widehat{\Phi}}
\newcommand{\hPsi}{\widehat{\Psi}}
\newcommand{\hOmega}{\widehat{\Omega}}
\newcommand{\hXi}{\widehat{\Xi}}
\newcommand{\hLambda}{\widehat{\Lambda}}

\newcommand{\hbalpha}{\widehat{\balpha}}
\newcommand{\hbbeta}{\widehat{\bbeta}}
\newcommand{\hbgamma}{\widehat{\bgamma}}
\newcommand{\hbdelta}{\widehat{\bdelta}}
\newcommand{\hbtheta}{\widehat{\btheta}}
\newcommand{\hbmu}{\widehat{\bmu}}
\newcommand{\hbnu}{\widehat{\bnu}}
\newcommand{\hbtau}{\widehat{\btau}}
\newcommand{\hbpi}{\widehat{\bpi}}
\newcommand{\hbepsilon}{\widehat{\bepsilon}}
\newcommand{\hbvarepsilon}{\widehat{\bvarepsilon}}
\newcommand{\hbsigma}{\widehat{\bsigma}}
\newcommand{\hbvarsigma}{\widehat{\bvarsigma}}
\newcommand{\hbzeta}{\widehat{\bzeta}}
\newcommand{\hbmeta}{\widehat{\beta}}
\newcommand{\hbkappa}{\widehat{\bkappa}}
\newcommand{\hbchi}{\widehat\bm{\latexchi}\movedownsub}
\newcommand{\hbphi}{\widehat{\bphi}}
\newcommand{\hbpsi}{\widehat{\bpsi}}
\newcommand{\hbomega}{\widehat{\bomega}}
\newcommand{\hbxi}{\widehat{\bxi}}
\newcommand{\hblambda}{\widehat{\blambda}}
\newcommand{\hbrho}{\widehat{\brho}}

\newcommand{\hbGamma}{\widehat{\bGamma}}
\newcommand{\hbDelta}{\widehat{\bDelta}}
\newcommand{\hbTheta}{\widehat{\bTheta}}
\newcommand{\hbPi}{\widehat{\bPi}}
\newcommand{\hbSigma}{\widehat{\bSigma}}
\newcommand{\hbPhi}{\widehat{\bPhi}}
\newcommand{\hbPsi}{\widehat{\bPsi}}
\newcommand{\hbOmega}{\widehat{\bOmega}}
\newcommand{\hbXi}{\widehat{\bXi}}
\newcommand{\hbLambda}{\widehat{\bLambda}}

\makeatletter
\newcommand*\rel@kern[1]{\kern#1\dimexpr\macc@kerna}
\newcommand*\widebar[1]{%
  \begingroup
  \def\mathaccent##1##2{%
    \rel@kern{0.8}%
    \overline{\rel@kern{-0.8}\macc@nucleus\rel@kern{0.2}}%
    \rel@kern{-0.2}%
  }%
  \macc@depth\@ne
  \let\math@bgroup\@empty \let\math@egroup\macc@set@skewchar
  \mathsurround\z@ \frozen@everymath{\mathgroup\macc@group\relax}%
  \macc@set@skewchar\relax
  \let\mathaccentV\macc@nested@a
  \macc@nested@a\relax111{#1}%
  \endgroup
}
\makeatother

\newcommand{\barbalpha}{\widebar{\balpha}}
\newcommand{\barbbeta}{\widebar{\bbeta}}
\newcommand{\barbgamma}{\widebar{\bgamma}}
\newcommand{\barbdelta}{\widebar{\bdelta}}
\newcommand{\barbtheta}{\widebar{\btheta}}
\newcommand{\barbmu}{\widebar{\bmu}}
\newcommand{\barbnu}{\widebar{\bnu}}
\newcommand{\barbtau}{\widebar{\btau}}
\newcommand{\barbpi}{\widebar{\bpi}}
\newcommand{\barbepsilon}{\widebar{\bepsilon}}
\newcommand{\barbvarepsilon}{\widebar{\bvarepsilon}}
\newcommand{\barbsigma}{\widebar{\bsigma}}
\newcommand{\barbvarsigma}{\widebar{\bvarsigma}}
\newcommand{\barbzeta}{\widebar{\bzeta}}
\newcommand{\barbmeta}{\widebar{\beta}}
\newcommand{\barbkappa}{\widebar{\bkappa}}
\newcommand{\barbchi}{\bar\bm{\latexchi}\movedownsub}
\newcommand{\barbphi}{\widebar{\bphi}}
\newcommand{\barbpsi}{\widebar{\bpsi}}
\newcommand{\barbomega}{\widebar{\bomega}}
\newcommand{\barbxi}{\widebar{\bxi}}
\newcommand{\barblambda}{\widebar{\blambda}}
\newcommand{\barbrho}{\widebar{\brho}}

\newcommand{\barbGamma}{\widebar{\bGamma}}
\newcommand{\barbDelta}{\widebar{\bDelta}}
\newcommand{\barbTheta}{\widebar{\bTheta}}
\newcommand{\barbPi}{\widebar{\bPi}}
\newcommand{\barbSigma}{\widebar{\bSigma}}
\newcommand{\barbPhi}{\widebar{\bPhi}}
\newcommand{\barbPsi}{\widebar{\bPsi}}
\newcommand{\barbOmega}{\widebar{\bOmega}}
\newcommand{\barbXi}{\widebar{\bXi}}
\newcommand{\barbLambda}{\widebar{\bLambda}}

%MathOperator
\DeclareMathOperator*{\argmax}{arg\,max}
\DeclareMathOperator*{\argmin}{arg\,min}
\DeclareMathOperator*{\argsup}{arg\,sup}
\DeclareMathOperator*{\arginf}{arg\,inf}
\DeclareMathOperator*{\minimize}{minimize}
\DeclareMathOperator*{\maximize}{maximize}
\DeclareMathOperator{\ST}{s.t.\ }
%\DeclareMathOperator{\ST}{subject\,\,to}
\DeclareMathOperator{\as}{a.s.}
\DeclareMathOperator{\const}{const}
\DeclareMathOperator{\diag}{diag}
\DeclareMathOperator{\cum}{cum}
\DeclareMathOperator{\sgn}{sgn}
\DeclareMathOperator{\tr}{tr}
\DeclareMathOperator{\Tr}{Tr}
\DeclareMathOperator{\spn}{span}
\DeclareMathOperator{\supp}{supp}
\DeclareMathOperator{\adj}{adj}
\DeclareMathOperator{\var}{var}
\DeclareMathOperator{\Vol}{Vol}
\DeclareMathOperator{\cov}{cov}
\DeclareMathOperator{\corr}{corr}
\DeclareMathOperator{\sech}{sech}
\DeclareMathOperator{\sinc}{sinc}
\DeclareMathOperator{\rank}{rank}
\DeclareMathOperator{\poly}{poly}
\DeclareMathOperator{\vect}{vec}
\DeclareMathOperator{\conv}{conv}
\DeclareMathOperator*{\lms}{l.i.m.\,}
\DeclareMathOperator*{\esssup}{ess\,sup}
\DeclareMathOperator*{\essinf}{ess\,inf}
\DeclareMathOperator{\sign}{sign}
\DeclareMathOperator{\eig}{eig}
\DeclareMathOperator{\ima}{im}
\DeclareMathOperator{\Mod}{mod}
\DeclareMathOperator*{\concat}{\scalerel*{\parallel}{\sum}}


%Paired delimiters
\newcommand{\ifbcdot}[1]{\ifblank{#1}{\cdot}{#1}}

\DeclarePairedDelimiterX\abs[1]{\lvert}{\rvert}{\ifbcdot{#1}}
\DeclarePairedDelimiterX\parens[1]{(}{)}{\ifbcdot{#1}}
\DeclarePairedDelimiterX\brk[1]{[}{]}{\ifbcdot{#1}}
\DeclarePairedDelimiterX\braces[1]{\{}{\}}{\ifbcdot{#1}}
\DeclarePairedDelimiterX\angles[1]{\langle}{\rangle}{\ifblank{#1}{\cdot,\cdot}{#1}}
\DeclarePairedDelimiterX\ip[2]{\langle}{\rangle}{\ifbcdot{#1},\ifbcdot{#2}}
\DeclarePairedDelimiterX\norm[1]{\lVert}{\rVert}{\ifbcdot{#1}}
\DeclarePairedDelimiterX\ceil[1]{\lceil}{\rceil}{\ifbcdot{#1}}
\DeclarePairedDelimiterX\floor[1]{\lfloor}{\rfloor}{\ifbcdot{#1}}

% Math symbol font matha
\DeclareFontFamily{U}{matha}{\hyphenchar\font45}
\DeclareFontShape{U}{matha}{m}{n}{
      <5> <6> <7> <8> <9> <10> gen * matha
      <10.95> matha10 <12> <14.4> <17.28> <20.74> <24.88> matha12
      }{}
\DeclareSymbolFont{matha}{U}{matha}{m}{n}
\DeclareFontSubstitution{U}{matha}{m}{n}

% Math symbol font mathb
\DeclareFontFamily{U}{mathx}{\hyphenchar\font45}
\DeclareFontShape{U}{mathx}{m}{n}{
      <5> <6> <7> <8> <9> <10>
      <10.95> <12> <14.4> <17.28> <20.74> <24.88>
      mathx10
      }{}
\DeclareSymbolFont{mathx}{U}{mathx}{m}{n}
\DeclareFontSubstitution{U}{mathx}{m}{n}

% Symbol definition
\DeclareMathDelimiter{\vvvert}{0}{matha}{"7E}{mathx}{"17}
\DeclarePairedDelimiterX\vertiii[1]{\vvvert}{\vvvert}{\ifbcdot{#1}}

\DeclarePairedDelimiterXPP\trace[1]{\operatorname{Tr}}{(}{)}{}{\ifbcdot{#1}} % column vector
\DeclarePairedDelimiterXPP\col[1]{\operatorname{col}}{\{}{\}}{}{\ifbcdot{#1}} % column vector
\DeclarePairedDelimiterXPP\row[1]{\operatorname{row}}{\{}{\}}{}{\ifbcdot{#1}} % row vector
\DeclarePairedDelimiterXPP\erf[1]{\operatorname{erf}}{(}{)}{}{\ifbcdot{#1}}
\DeclarePairedDelimiterXPP\erfc[1]{\operatorname{erfc}}{(}{)}{}{\ifbcdot{#1}}
\DeclarePairedDelimiterXPP\KLD[2]{D}{(}{)}{}{\ifbcdot{#1}\, \delimsize\|\, \ifbcdot{#2}} % KL divergence
\DeclarePairedDelimiterXPP\op[2]{\operatorname{#1}}{(}{)}{}{#2} % general operator

% Math relations
\newcommand{\convp}{\stackrel{\mathrm{p}}{\longrightarrow}}
\newcommand{\convas}{\stackrel{\mathrm{a.s.}}{\longrightarrow}}
\newcommand{\convd}{\stackrel{\mathrm{d}}{\longrightarrow}}
\newcommand{\convD}{\stackrel{\mathrm{D}}{\longrightarrow}}

\newcommand{\dotleq}{\stackrel{.}{\leq}}
\newcommand{\dotlt}{\stackrel{.}{<}}
\newcommand{\dotgeq}{\stackrel{.}{\geq}}
\newcommand{\dotgt}{\stackrel{.}{>}}
\newcommand{\dotdoteq}{\stackrel{\,..}{=}}

\newcommand{\eqa}[1]{\stackrel{#1}{=}}
\newcommand{\ed}{\eqa{\mathrm{d}}}
\newcommand{\lea}[1]{\stackrel{#1}{\le}}
\newcommand{\gea}[1]{\stackrel{#1}{\ge}}

\newcommand{\T}{^{\mkern-1.5mu\mathop\intercal}}% transpose notation
\newcommand{\Herm}{^{\mkern-1.5mu\mathsf{H}}}% Hermitian transpose notation
\newcommand{\setcomp}{^{\mathsf{c}}} %set complement
\newcommand{\ud}{\,\mathrm{d}} % for integrals like \int f(x) \ud x
\newcommand{\Id}{\mathrm{Id}} % identity function
\newcommand{\Bigmid}{{\ \Big| \ }}
\newcommand{\bzero}{\bm{0}}
\newcommand{\bone}{\bm{1}}

% Math functions
\DeclarePairedDelimiterXPP\indicate[1]{{\bf 1}}{\{}{\}}{}{\ifbcdot{#1}}
\newcommand{\indicator}[1]{{\bf 1}_{\braces*{\ifbcdot{#1}}}}
\newcommand{\indicatore}[1]{{\bf 1}_{#1}}
\newcommand{\tc}[1]{^{(#1)}}
\NewDocumentCommand\ofrac{s m}{%
	\IfBooleanTF#1%
	{\dfrac{1}{#2}}%
	{\frac{1}{#2}}%
}
\NewDocumentCommand\ddfrac{s m m}{%
	\IfBooleanTF#1%
	{\dfrac{\mathrm{d} {#2}}{\mathrm{d} {#3}}}%
	{\frac{\mathrm{d} {#2}}{\mathrm{d} {#3}}}%
}
\NewDocumentCommand\ppfrac{s m m}{%
	\IfBooleanTF#1%
	{\dfrac{\partial {#2}}{\partial {#3}}}%
	{\frac{\partial {#2}}{\partial {#3}}}%
}

\newcommand{\bmat}[1]{\begin{bmatrix} #1 \end{bmatrix}}
\newcommand{\smat}[1]{\left[\begin{smallmatrix} #1 \end{smallmatrix}\right]}

\newcommand{\Lh}[1]{\ell_{#1}}
\newcommand{\LLh}[1]{\log{\Lh{#1}}}

% just to make sure it exists
\providecommand\given{}
% can be useful to refer to this outside \set
\newcommand\SetSymbol[2][]{%
	\nonscript\, #1#2
	\allowbreak
	\nonscript\,
	\mathopen{}}

\DeclarePairedDelimiterX\Set[2]\{\}{%
\renewcommand\given{\SetSymbol[\delimsize]{#1}}
#2
}
\DeclarePairedDelimiterX\Setc[1]\{\}{%
\renewcommand\given{\SetSymbol{:}}
#1
}

% \set{x \given f(x)=1} gives \{x : f(x)=1\}
% \set[\vert]{x \given f(x)=1} gives \{x \vert f(x)=1\}
% Starred version uses \left and \right
\NewDocumentCommand\set{s o m}{%
	\IfBooleanTF#1%
	{\IfValueTF{#2}{\Set*{#2}{#3}}{\Setc*{#3}}}%
	{\IfValueTF{#2}{\Set{#2}{#3}}{\Setc{#3}}}%
}

%\NewDocumentCommand\set{s m t| m}{%
%\IfBooleanTF#1%
%{\left\{\, #2\mathrel{} \IfBooleanTF{#3}{\middle|}{:}\mathrel{}  #4\, \right\}}%
%{\{\, #2 \IfBooleanTF{#3}{\mid}{\mathrel{} : \mathrel{}} #4\, \}}% 
%}

\NewDocumentCommand{\evalat}{ s O{\big} m e{_^} }{%
\IfBooleanTF{#1}%
{\left. #3 \right|}{#3#2|}%
\IfValueT{#4}{_{#4}}%
\IfValueT{#5}{^{#5}}%
}


%%%%%%%%%%%%%%%%%%%%%%%%%%%%%%%%%%%%%%%%%%%%%%%%%%%%%%%%
% \P and \E simplified to remove @|, use \given instead.

\providecommand\given{}
\DeclarePairedDelimiterXPP\cprob[1]{}(){}{
\renewcommand\given{\nonscript\,\delimsize\vert\allowbreak\nonscript\,\mathopen{}}%
#1%
}
\DeclarePairedDelimiterXPP\cexp[1]{}[]{}{
\renewcommand\given{\nonscript\,\delimsize\vert\allowbreak\nonscript\,\mathopen{}}%
#1%
}

% Allows the use of 
% \P : \mathbb{P}
% \P(X) : \mathbb{P}\left({X}\right)
% \P_{p}(X) : \mathbb{P}_{p}\left({X}\right)
% \P(X \given Y) : \mathbb{P}\left({X}\, \middle| \, {Y}\right). 
% Starred version \P* does not use \left and \right. Maybe used in inline equations. 
\DeclareDocumentCommand \P { s e{_^} d() g } {%
	\mathbb{P}%
	\IfBooleanTF{#1}%
		{
			\IfValueT{#2}{_{#2}}%
			\IfValueT{#3}{^{#3}}%
			\IfValueTF{#5}{\cprob{#4 \given #5}}{\IfValueT{#4}{\cprob{#4}}}%
		}%
		{
			\IfValueT{#2}{_{#2}}%
			\IfValueT{#3}{^{#3}}%
			\IfValueTF{#5}{\cprob*{#4 \given #5}}{\IfValueT{#4}{\cprob*{#4}}}%
		}%
}

% Allows the use of 
% \E : \mathbb{E}
% \E[X] : \mathbb{E}\left[{X}\right]
% \E_{p}[X] or \E{p}[X] : \mathbb{E}_{p}\left[{X}\right]
% \E[X \given Y]: \mathbb{E}\left[{X}\, \middle| \, {Y}\right]. 
% Starred version \E* does not use \left and \right. Maybe used in inline equations. 
\DeclareDocumentCommand \E { s e{_^} o g } {%
	\mathbb{E}%
	\IfBooleanTF{#1}%
		{
			\IfValueT{#2}{_{#2}}%
			\IfValueT{#3}{^{#3}}%
			\IfValueTF{#5}{\cexp{#4 \given #5}}{\IfValueT{#4}{\cexp{#4}}}%
		}%
		{
			\IfValueT{#2}{_{#2}}%
			\IfValueT{#3}{^{#3}}%	
			\IfValueTF{#5}{\cexp*{#4 \given #5}}{\IfValueT{#4}{\cexp*{#4}}}%		
			%\IfValueT{#4}{\cexp*{#4}}%
		}%
}


\DeclareDocumentCommand \Var { s e{_^} d() g } {%
	\var%
	\IfBooleanTF{#1}%
		{
			\IfValueT{#2}{_{#2}}%
			\IfValueT{#3}{^{#3}}%
			\IfValueTF{#5}{\cprob{#4 \given #5}}{\IfValueT{#4}{\cprob{#4}}}%
		}%
		{
			\IfValueT{#2}{_{#2}}%
			\IfValueT{#3}{^{#3}}%	
			\IfValueTF{#5}{\cprob*{#4 \given #5}}{\IfValueT{#4}{\cprob*{#4}}}%		
			%\IfValueT{#4}{\cprob*{#4}}%
		}%
}

\DeclareDocumentCommand \Cov { s e{_^} d() g } {%
	\cov%
	\IfBooleanTF{#1}%
		{
			\IfValueT{#2}{_{#2}}%
			\IfValueT{#3}{^{#3}}%
			\IfValueTF{#5}{\cprob{#4 \given #5}}{\IfValueT{#4}{\cprob{#4}}}%
		}%
		{
			\IfValueT{#2}{_{#2}}%
			\IfValueT{#3}{^{#3}}%	
			\IfValueTF{#5}{\cprob*{#4 \given #5}}{\IfValueT{#4}{\cprob*{#4}}}%		
			%\IfValueT{#4}{\cprob*{#4}}%
		}%
}


% General distribution 
% E.g., \dist{Beta}[a,b][x] gives Beta(x | a,b); \dist{Beta}[a,b] gives Beta(a,b)
\ExplSyntaxOn
\NewDocumentCommand \dist {m o o} {%
\mathrm{#1}\left(%
	\IfValueT{#3}{%
		\tl_if_blank:nTF{ #3 }{\cdot\, \middle|\, }{#3\, \middle|\, }%
	}
	\IfValueT{#2}{#2}%
\right)%
}
\ExplSyntaxOff

\newcommand{\Bern}[1]{\dist{Bern}[#1]}
\newcommand{\Unif}[1]{\dist{Unif}[#1]}
\newcommand{\Dir}[1]{\dist{Dir}[#1]}
\newcommand{\Cat}[1]{\dist{Cat}[#1]}
\newcommand{\N}[2]{\dist{\calN}[#1,\, #2]}
\newcommand{\Beta}[2]{\dist{Beta}[#1,\, #2]}

\def\indep#1#2{\mathrel{\rlap{$#1#2$}\mkern5mu{#1#2}}}
\newcommand{\independent}{\protect\mathpalette{\protect\indep}{\perp}}


%Misc

% Colored underbrace/overbrace: 
% \cbrace[blue](5em){text}{text in underbrace}
% \cbrace+[blue](5em){text}{text in overbrace}
\NewDocumentCommand {\cbrace} {t+ D[]{black} D(){\widthof{#5}} m m } {%
	\begingroup%
		\color{#2}
		\IfBooleanTF{#1}{%
			\overbrace{#4}^%
		}{
			\underbrace{#4}_%
		}%
		{\parbox[c]{#3}{\centering\footnotesize{#5}}}%
	\endgroup% 
}

\let\oldforall\forall
\renewcommand{\forall}{\oldforall \, }

\let\oldexist\exists
\renewcommand{\exists}{\oldexist \, }

\newcommand\existu{\oldexist! \, }

% Tables
\makeatletter
\newcommand{\udcloser}[1]{\underline{\smash{#1}}}
\newcommand{\sd}[1]{{\scriptstyle \pm #1}}
\newcommand{\rankcolor}[2]{%
	\expandafter\renewcommand\csname #1\endcsname[1]{%
		\ifblank{##1}{%
			{\color{#2} \textbf{#2}}%
		}{%
			\ifmmode
				\textcolor{#2}{\bm{##1}}%
			\else%
				{\color{#2} \textbf{##1}}%
			\fi	
		}%
	}
}

\providecommand{\first}{}
\providecommand{\second}{}
\providecommand{\third}{}

% You can redefine these using \rankcolor in your manuscript if necessary.
\rankcolor{first}{red}
\rankcolor{second}{blue}
\rankcolor{third}{cyan}
\makeatother

% Figures
%\renewcommand{\figurename}{Fig.}
\newcommand{\figref}[1]{Fig.~\ref{#1}}
\graphicspath{{./Figures/}{./figures/}}
\pdfsuppresswarningpagegroup=1

% Need to enable write18 to use this.
\DeclareDocumentCommand{\includeCroppedPdf}{ o O{./Figures/} m }{
	\IfFileExists{#2#3-crop.pdf}{}{%
		\immediate\write18{pdfcrop #2#3.pdf #2#3-crop.pdf}}%
	\includegraphics[#1]{#2#3-crop.pdf}
}


%%%%%%%%%%%%%%%%%%%%%%%%%%%%%%%%%%%%%%%%%%%%%%%%%%%%%%%%%%%%%%%%%%%%%%%%%

% Supplement
\newcommand{\beginsupplement}{
	\setcounter{section}{0}
	\renewcommand{\thesection}{S\arabic{section}}
	\renewcommand{\thesectiondis}{S\arabic{section}.}
	\setcounter{equation}{0}
	\renewcommand{\theequation}{S\arabic{equation}}
	\setcounter{table}{0}
	\renewcommand{\thetable}{S\arabic{table}}
	\setcounter{figure}{0}
	\renewcommand{\thefigure}{S\arabic{figure}}
}

% Use with xr-hyper
\makeatletter
\newcommand*{\addFileDependency}[1]{% argument=file name and extension
  \typeout{(#1)}
  \@addtofilelist{#1}
  \IfFileExists{#1}{}{\typeout{No file #1.}}
}
\makeatother

\newcommand*{\myexternaldocument}[1]{%
    \externaldocument{#1}%
    \addFileDependency{#1.tex}%
    \addFileDependency{#1.aux}%
}

% Editing
\definecolor{gray90}{gray}{0.9}
\def\colorlist{red,blue,brown,cyan,darkgray,gray,lightgray,green,lime,magenta,olive,orange,pink,purple,teal,violet,white,yellow}

% Define the \createcolor macro
\makeatletter
\def\startcomment{[}
\ifx\nohighlights\undefined
	\newcommand{\createcolor}[1]{%
			\expandafter\newcommand\csname #1\endcsname[1]{{\color{#1} ##1}}%
	}
	\newcommand{\msout}[1]{\text{\color{green} \sout{\ensuremath{#1}}}}
	\newcommand{\del}[1]{{\color{green}\ifmmode \msout{#1}\else\sout{#1}\fi}}
\else
	\newcommand{\createcolor}[1]{%
			\expandafter\newcommand\csname #1\endcsname[1]{%
				\noexpandarg%
				\StrChar{##1}{1}[\firstletter]%
				\if\firstletter\startcomment%
					\relax
				\else%
					##1
				\fi
			}%
	}
	\newcommand{\msout}[1]{}
	\newcommand{\del}[1]{}
\fi

\def\@tempa#1,{%
    \ifx\relax#1\relax\else
        \createcolor{#1}%
        \expandafter\@tempa
    \fi
}
\expandafter\@tempa\colorlist,\relax,
\makeatother

\newcommand{\old}[1]{{\color{green} [\textrm{DELETED: }#1]}}
\newcommand{\hhide}[1]{}
%\newcommand{\hhide}[1]{{\color{magenta} [TO BE EXCLUDED] #1}}

\newcommand{\txp}[2]{\texorpdfstring{#1}{#2}}

%%%%%%%%%%%%%%%%%%%%%%%%%%%%%%%%%%%%%%%%%%%%%%%%%
% For diagnosis: if activated, will show what is causing 
% LaTeX Warning: Label(s) may have changed. Rerun to get cross-references right.

\ifx\diagnoselabel\undefined
	\relax
\else
	\makeatletter
	\def\@testdef #1#2#3{%
		\def\reserved@a{#3}\expandafter \ifx \csname #1@#2\endcsname
			\reserved@a  \else
			\typeout{^^Jlabel #2 changed:^^J%
				\meaning\reserved@a^^J%
				\expandafter\meaning\csname #1@#2\endcsname^^J}%
			\@tempswatrue \fi}
	\makeatother
\fi

%%%%%%%%%%%%%%%%%%%%%%%%%%%%%%%%%%%%%%%%%%%%%%%%%%

\def\UrlFont{\tt}


\newcounter{week}

%Theorem declarations
\newtheorem{Theorem}{Theorem}[week]
\newtheorem{Corollary}{Corollary}[week]
\newtheorem{Proposition}{Proposition}[week]
\newtheorem{Lemma}{Lemma}[week]
\newtheorem{Definition}{Definition}[week]
\newtheorem{Assumption}{Assumption}[week]

%\theoremstyle{definition}
\newtheorem{Example}{Example}[week]
\newtheorem{Remark}{Remark}[week]
\newtheorem{Exercise}{Exercise}[week]


\newcommand{\handout}[2]{
	\setcounter{week}{#1}
  \noindent
  \begin{center}
  \framebox{
    \vbox{
      \hbox to 6in {\bf An Analytical Introduction to Probability Theory \hfill}
      \vspace{5mm}
      \hbox to 6in { {\Large \hfill #1.~#2  \hfill} }
      \vspace{5mm}
      \hbox to 6in { {\em DASN, NTU \hfill \small{\url{https://personal.ntu.edu.sg/wptay/}}}}
    }
  }
  \end{center}
	\renewcommand{\thesection}{{#1}.\arabic{section}}
  \vspace*{4mm}
}

\newcommand{\calBR}{\calB(\Real)}
\newcommand{\io}{\ \mathrm{i.o.}}
\newcommand{\fo}{\ \mathrm{f.o.}}

\externaldocument{1_BasicRealAnalysis_I}
\externaldocument{2_BasicRealAnalysis_II}

\begin{document}

\handout{3}{Probability Spaces}


\section{Introduction}

Recommended reference: ``Probability: Theory and Examples'' by Rick Durrett.

Let $\Omega$ be a sample space. An event is a subset of $\Omega$. We are interested to define a ``likelihood'' or ``chance'' for each event to happen in the future. We call this the probability of the event.
 
\begin{Example}\label{wk3:example1}
Let $\Omega = [0,1]$, the probability of the event $(a, b]$, where $0\leq a \leq b < 1$ can be defined by
\begin{align*}
\P({(a, b]}) = F(b) - F(a),
\end{align*}
where $F$ is a non-decreasing and right-continuous (we will see later why this is needed) function  with
\begin{align*}
&\lim_{x \to 0} F(x) = 0, \\
&\lim_{x \to 1} F(x) = 1.
\end{align*}
However, there are many other events like $\bigcup\limits_{i=1}^{\infty} (a_i,b_i]$ whose probabilities we are interested in. In particular, $\P$ should have the following properties:
\begin{enumerate}[(i)]
	\item\label{prop3} $\P(\Omega)=1$.
	\item\label{prop1} If $A_1, A_2, \ldots$ are disjoint sets, then
		\begin{align*}
		\P(\bigcup_{i\geq1} A_i) = \sum_{i\geq1} \P(A_i).
		\end{align*}
	\item\label{prop2} If $A$ is congruent to $B$ (i.e., $A$ is $B$ transformed by translation, rotation or reflection), then $\P(A)=\P(B)$.
\end{enumerate}
Unfortunately, for these conditions to hold for \emph{all} events would lead to inconsistency. To see why, define an equivalence $x \sim y$ iff $x - y$ is rational. Then $\Omega$ can be partitioned into equivalence classes. Let $N \subset \Omega$ be a subset that contains exactly one member of each equivalence class (we need the axiom of choice here). For each rational number $r \in \bbQ\cap [0,1)$, let
\begin{align*}
N_r = \set*{x+r \given x \in N\cap [0,1-r)} \cup \set*{x+r-1 \given x \in N\cap [1-r,1]},
\end{align*}
i.e., $N_r$ is $N$ translated to the right by $r$ with the part after $[0,1)$ shifted to the front (wrapped around) so that $N_r \subset \Omega=[0,1]$. From properties~\ref{prop1} and \ref{prop2}, we have for any rational $r \in\bbQ\cap [0,1)$,
\begin{align}\label{eq:N_r}
\P(N) = \P(N\cap [0,1-r)) + \P(N\cap [1-r,1)) = \P(N_r).
\end{align}

We also have the following:
\begin{enumerate}
	\item Every $x\in\Omega$ belongs to a $N_r$ because if $y\in N$ is an element of the equivalence class of $x$, then $x \in N_r$ where $r = x-y$ if $x\geq y$ or $r=x-y+1$ if $x<y$. 
	\item Every $x\in\Omega$ belongs to exactly one $N_r$ because if $x\in N_r\cap N_s$ for $r\ne s$, then $x-r$ or $x-r+1$ and $x-s$ or $x-s+1$ would be distinct elements of $N$ belonging to the same equivalence class, contradicting how we chose $N$. 
\end{enumerate}
Therefore, $\Omega$ is the disjoint union of $N_r$ over all rational $r\in \bbQ\cap [0,1)$. From properties~\ref{prop3} and \ref{prop1}, we also have $1 = \P(\Omega) = \sum_r \P(N_r)$. But $\P(N_r)=\P(N)$ from \cref{eq:N_r}, so the sum is either $0$ if $\P(N)=0$ or $\infty$ if $\P(N)>0$, a contradiction. 

This example shows that it is impossible to define a suitable $\P$ for all possible events, some of which are very weird objects (Banach and Tarski (1924) showed that in $\Real^n$ where $n\geq3$, even stranger subsets can be constructed!). The solution that mathematicians have come up with is to restrict to a collection of subsets \emph{and} a $\P$ with ``nice'' properties, i.e., a $\sigma$-algebra and measure, respectively. 
\end{Example}

\section{\txp{$\sigma$}{Sigma}-algebras and Measures}
Let $\calA$ be a collection of events (collection of subsets of $\Omega$). 

\begin{Definition}
$\calA$ is an algebra if
\begin{enumerate}[(i)]
\item $\Omega \in \calA$.
\item $A \in \calA \implies A^c = \Omega \setminus A \in \calA$.
\item $A_1, A_2 \in \calA \implies A_1 \cup A_2 \in \calA$. By induction, $A_i \in \calA, \forall i = 1, \ldots, n, \implies \bigcup\limits_{i=1}^{n} A_i \in \calA$.
\end{enumerate}
\end{Definition}

\begin{Definition}
$\calA$ is a $\sigma$-algebra or $\sigma$-field if
\begin{enumerate}[(i)]
\item $\Omega \in \calA$.
\item $A \in \calA \implies A^c = \Omega \setminus A \in \calA$.
\item $A_i \in \calA, \forall i = 1, 2, \ldots \implies \bigcup\limits_{i=1}^{\infty} A_i \in \calA$.
\end{enumerate}
\end{Definition}

$\left(\Omega, \calA\right)$ is called a measurable space if $\calA$ is a $\sigma$-algebra. A set $A \in \calA$ is said to be measurable.

\begin{Definition}
For a measurable space $(\Omega,\calA)$, a function $\P: \calA \mapsto \left[0, 1\right]$ is a probability measure if
\begin{enumerate}[(i)]
\item $\P(\Omega) = 1$.
\item $A_1, A_2, \ldots \in \calA$ with $A_i \cap A_j = \emptyset$, $\forall i \neq j$ $\implies \P(\bigcup\limits_{i=1}^{\infty}A_i) = \sum\limits_{i=1}^{\infty}\P(A_i)$ (countably additive).
\end{enumerate}
\end{Definition}

Let $\calA$ be a $\sigma$-algebra.

\begin{Lemma} \label{wk3:lem:increasing_seq}
Suppose $B_i \in \calA, B_i \subset B_{i+1}, \forall i\geq 1$, then
\begin{align*}
\P(\bigcup\limits_{i=1}^{\infty} B_i) = \lim_{i \to \infty}\P(B_i)
\end{align*}
\end{Lemma}
\begin{proof}
Let $C_1 = B_1, C_i = B_i \cap B_{i-1}^c, \forall i \geq 2$, then the $C_i$'s are disjoint, and we have
\begin{align*}
B_n &= \bigcup\limits_{i=1}^{n} C_i, \\
\bigcup\limits_{i=1}^{\infty} B_i &= \bigcup\limits_{i=1}^{\infty} C_i.
\end{align*}
Then we obtain
\begin{align*}
& \P(\bigcup\limits_{i=1}^{\infty} B_i) 
= \P(\bigcup\limits_{i=1}^{\infty} C_i) 
= \sum\limits_{i=1}^{\infty} \P(C_i) 
= \lim_{n \to \infty} \sum\limits_{i=1}^{n} \P(C_i)
= \lim_{n \to \infty} \P(\bigcup\limits_{i=1}^{n} C_i) 
= \lim_{n \to \infty} \P(B_n).
\end{align*}
\end{proof}

\begin{Corollary}
For $A_i \in \calA, \forall\ i=1, 2, \ldots$, we have
\begin{align*}
\lim_{n \to \infty} \P(\bigcup\limits_{i=1}^{n} A_i) = \P(\bigcup\limits_{i=1}^{\infty} A_i).
\end{align*}
\end{Corollary}
\begin{proof}
Let $B_n = \bigcup\limits_{i=1}^n A_i$, which is an increasing sequence. We have $\P(\bigcup\limits_{i=1}^{\infty} A_i) = \P(\bigcup\limits_{n=1}^{\infty} B_n)$. From \cref{wk3:lem:increasing_seq}, we obtain $\P(\bigcup\limits_{n=1}^{\infty} B_n) = \lim\limits_{n \to \infty}\P(B_n) = \lim\limits_{n \to \infty} \P(\bigcup\limits_{i=1}^{n} A_i)$.
\end{proof}

\begin{Corollary}\label{wk3:cor:dec_seq}
For a decreasing sequence $B_i \supset B_{i+1}, \forall\ i\geq1$, we have
\begin{align*}
\P(\bigcap\limits_{i=1}^{\infty} B_i) = \lim_{i \to \infty} \P(B_i).
\end{align*}
\end{Corollary}
%
\begin{proof}
Similar to the proof of \cref{wk3:lem:increasing_seq}.
\end{proof}

\begin{Lemma}
For $A, B \in \calA, A \subset B$, we have $\P(A) \leq \P(B)$.
\end{Lemma}
\begin{proof}
$B = A \cup (B \setminus A) \implies \P(B) = \P(A) + \P(B \setminus A) \geq \P(A)$.
\end{proof}

\begin{Lemma}[Union bound]\label{wk3:lem:union_bound}
For $A_1, A_2, \ldots \in \calA$, we have 
\begin{align*}
\P(\bigcup_{i\geq1} A_i) \leq \sum_{i\geq1} \P(A_i).
\end{align*}
\end{Lemma}
\begin{proof}
Let $B_i = A_i \backslash \bigcup_{j<i} A_j$ for $i\geq 1$. Then the $B_i$'s are disjoint, $B_i \subset A_i$, $\bigcup_{i\geq1} A_i =  \bigcup_{i\geq1} B_i$ and
\begin{align*}
\P(\bigcup_{i\geq1} A_i) 
&= \P(\bigcup_{i\geq1} B_i)\\ 
&= \sum_{i\geq1} \P(B_i)\\
&\leq \sum_{i\geq1} \P(A_i).
\end{align*}
\end{proof}


In \cref{wk3:example1}, let 
\begin{align*}
\calA' = \set*{\bigcup\limits_{i=1}^{n} \left(a_i, b_i\right] \given n \geq 1, \left(a_i, b_i\right] \subset (0, 1], \left(a_i, b_i\right] \cap \left(a_j, b_j\right] = \emptyset, \forall\ i \neq j}. 
\end{align*}
Check that $\calA'$ is an algebra. For each element of $\calA'$, we define
\begin{align*}
\P(\bigcup\limits_{i=1}^{n} {(a_i, b_i]}) &= \sum_{i=1}^n \P({(a_i, b_i]})
\end{align*}
for each $n\geq1$. One can show that with this definition, $\P$ is countably additive on $\calA'$, i.e., whenever $A_i \in \calA'$, $i\geq 1$ and $\bigcup_{i\geq 1} A_i \in \calA$ are finite unions of disjoint intervals, we have
\begin{align*}
\P(\bigcup\limits_{i\geq 1} {(a_i, b_i]}) &= \sum_{i\geq 1} \P({(a_i, b_i]}).
\end{align*}
We also have
\begin{align*}
\P({(a, b]}) 
&= \P(\bigcap\limits_{n=1}^{\infty} {(a, b + 1/n ]}) \\
&= \lim_{n \to \infty} \P({(a, b + 1/n]}) \\
&= \lim_{n \to \infty} \left(F(b+1/n) - F(a) \right) \\
&= F(b) - F(a),
\end{align*}
where the second equality follows from \cref{wk3:cor:dec_seq} and the last equality requires the right-continuity of $F$. 

Let $\calA = \sigma(\calA')$ be the $\sigma$-algebra generated by $\calA'$, i.e., the intersection of all $\sigma$-algebras that contain $\calA'$. Note that this is well-defined as a trivial $\sigma$-algebra containing $\calA'$ is the power set of $\Omega$. It is easy to show that $\calA$ is the smallest $\sigma$-algebra containing $\calA'$.

\begin{Theorem}\label{Caratheodory Theorem}
Carath{\'e}odory's Extension Theorem. If $\calA'$ is an algebra, $\P : \calA' \mapsto [0, 1]$ is countably additive on $\calA'$ and $\P(\emptyset)=0$, then $\P$ has a unique extension to $\calA = \sigma(\calA')$.
\end{Theorem}
%

The proof of the existence of such an extension $\P: \calA \mapsto [0,1]$ can be found in the book by Durrett. We focus on the proof of uniqueness here. We make use of the very useful Dynkin's $\pi$-$\lambda$ Theorem.

\begin{Definition}
$\calP$ is a $\pi$-system if $A, B \in \calP \implies A \cap B \in \calP$.
\end{Definition}
\begin{Definition}
$\calL$ is a $\lambda$-system if 
\begin{enumerate}[(i)]
\item $\Omega \in \calL$.
\item $A \in \calL \implies A^c = \Omega \setminus A \in \calL$.
\item $A_i \in \calL, \forall\ i\geq1,\ A_i \cap A_j = \emptyset, \forall\ i \neq j \implies \bigcup\limits_{i=1}^{\infty} A_i \in \calL$.
\end{enumerate}
\end{Definition}
\begin{Remark}
If $\calA$ is both a $\pi$-system and $\lambda$-system, then $\calA$ is $\sigma$-algebra.
\end{Remark}
%
\begin{Theorem}[Dynkin's $\pi$-$\lambda$ Theorem] \label{Thm:Dynkin_pi_lambda_thm}
If $\calP$ is a $\pi$-system and $\calL$ is a $\lambda$-system with $\calP \subset \calL$, then $\sigma(\calP) \subset \calL$.
\end{Theorem}
\begin{proof}
Let $\ell(\calP)$ be the smallest $\lambda$-system that contains $\calP$. If $\ell(\calP)$ is a $\pi$-system, $\ell(\calP)$ is a $\sigma$-algebra. Then we have
\begin{align*}
\sigma(\calP) \subset \ell(\calP) \subset \calL.
\end{align*}
Therefore, it suffices to prove that $\ell(\calP)$ is a $\pi$-system. We prove that $\ell(\calP)$ is a $\pi$-system in the following three steps. For any $A \subset \Omega$, let $\calG_A = \{B \subset \Omega : B \cap A \in \ell(\calP) \}$.

Step 1:  We show that if $A \in \ell(\calP)$, then $\calG_A$ is $\lambda$-system. This is done by checking the following conditions:
\begin{enumerate}[(i)]
\item $\Omega \cap A = A \in \ell(\calP) \implies \Omega \in \calG_A$.
\item Suppose $B \in \calG_A$. We have $B^c \cap A = \left( \left(B \cap A \right) \cup {A^c} \right)^c$. Then we have
\begin{align*}
B \cap A,\ A^c \in  \ell(\calP) \implies \left(B \cap A \right) \cup {A^c} \in \ell(\calP) \implies \left( \left(B \cap A \right) \cup {A^c} \right)^c \in \ell(\calP),
\end{align*}
which implies that $B^c \in \calG_A$.
\item Let $B_i \in \calG_A, \forall\ i = 1, 2, \ldots$ with $B_i \cap B_j = \emptyset, \forall\ i \neq j$. Therefore, $B_i \cap A \in \ell(\calP), \forall\ i = 1, 2, \ldots$ are also disjoint. Then we have
\begin{align*}
\left(\bigcup\limits_{i=1}^{\infty} B_i\right) \cap A = \bigcup\limits_{i=1}^{\infty} (B_i \cap A ) \in \ell(\calP) \implies \bigcup\limits_{i=1}^{\infty} B_i \in \calG_A.
\end{align*}
\end{enumerate}

Step 2: We show that if $B \in \calP \subset \ell(\calP)$, then $\ell(\calP) \subset \calG_B$. Since $\calP$ is a $\pi$-system, we have
\begin{align*}
\forall\ C \in \calP, C \cap B \in \calP \subset \ell(\calP),
\end{align*}
which means that 
\begin{align*}
\calP \subset \calG_B,
\end{align*}
where $\calG_B$ is a $\lambda$-system from Step 1 since $B \in \ell(\calP)$. Therefore, 
\begin{align} \label{Thm:interm_result2}
\ell(\calP) \subset \calG_B. 
\end{align}

Step 3: Consider any $B \in \calP$ and an $A \in \ell(\calP)$. From Step 2, we have $A \in \calG_B$. Therefore, $A \cap B \in \ell(\calP)$ and 
\begin{align*}
\calP \subset \calG_A,
\end{align*}
where $\calG_A$ is a $\lambda$-system from Step 1. Thus $\ell(\calP) \subset \calG_A$. This means that for any $C \in \ell(\calP)$, we have
\begin{align*}
C \in \calG_A \implies C \cap A \in \ell(\calP).
\end{align*}
Since $A, C \in \ell(\calP)$, the above result immediately shows that $\ell(\calP)$ is a $\pi$-system.
\end{proof}

We now return to the uniqueness proof of \cref{Caratheodory Theorem}. Suppose $\P_1$ and $\P_2$ are extensions of $\P$ with $\P_1(A) = \P_2(A), \forall\ A \in \calA'$. Let
\begin{align*}
\calL = \{A \in \calA: \P_1(A) = \P_2(A) \}.
\end{align*}
It is easy to see that $\calL$ is a $\lambda$-system due to the properties of probability measures. $\calA'$ is a $\pi$-system since it is an algebra, and $\calA' \subset \calL$ by definition. According to \cref{Thm:Dynkin_pi_lambda_thm}, $\calA = \sigma(\calA') \subset \calL$. Thus, $\P_1 = \P_2$ on $\calA$, meaning the extension is unique on $\calA$.

\section{Regularity}

\begin{Definition} \label{Defns:Borel_sigma_algebra}
Let $(\Omega, d)$ be a metric space. The Borel $\sigma$-algebra is a $\sigma$-algebra generated by the open sets of $\Omega$ (or equivalently, by the closed sets).
\end{Definition}
%
\begin{Lemma}\label{wk3:lem:closed_approx}
Let $\calB$ be the Borel $\sigma$-algebra. Then, $\forall A \in \calB$,
\begin{align} \label{closed_approx}
\P(A) = \sup \set*{ \P(F) \given F \subset A,\, F\ \text{is closed}}.
\end{align}
\end{Lemma}
\begin{proof}
Let
\begin{align*}
\calL = \set*{A \in \calB \given \text{both $A$ and $A^c$ satisfy \cref{closed_approx}}}.
\end{align*} 
It can be checked that $\calL$ is a $\lambda$-system (exercise). Let $F$ be closed. It is obvious that $F$ satisfies \cref{closed_approx}. Let $U = F^c$. We show that $U$ also satisfies \cref{closed_approx}. To do this, since $\sup \P(C) \leq \P(U)$ for all closed $C\subset U$, it suffices to show that there is a sequence of closed subsets $F_n\subset U$ such that $\P(U)=\sup_n \P(F_n)$. To this end, for $n\geq 1$, let
\begin{align*}
F_n = \left\{\omega \in \Omega: \min\limits_{x \in F}d(\omega, x) \geq 1/n \right\}.
\end{align*}
Then we have $F_n \subset F_{n+1}$ and $U = \bigcup\limits_{n=1}^{\infty} F_n$. From \cref{wk3:lem:increasing_seq}, we obtain
\begin{align*}
\P(U) = \lim_{n \to \infty} \P(F_n) = \sup\limits_{n \geq 1} \P(F_n).
\end{align*}
Therefore, $U$ satisfies \cref{closed_approx}. As a consequence, $F \in \calL$ for any closed $F$. Since $\calB$ is generated by the closed sets, we have $\calB \in \calL$ and the proof is complete.
\end{proof}

A metric space is separable if it has a countable dense subset, i.e., $\exists \{x_n\}_{n=1}^{\infty}$ such that $\forall \text{open}\ U \subset \Omega$, $x_i \in U$ for some $x_i$. Exercise: show that totally bounded implies separable. The converse is not true (e.g., consider the discrete metric space in \cref{ex:discrete_X}). 

\begin{Definition}\label{wk3:def:regular}
We say that a probability measure $\P$ for $(\Omega,\calB)$ is regular if
\begin{align*}
\P(A) = \sup\left\{\P(K): K \subset A, K\ \text{is compact}\right\}
\end{align*}
for all $A\in\calB$.
\end{Definition}

\begin{Theorem}[Ulam]\label{wk3:thm:Ulam}
If $(\Omega, d)$ is a complete separable space with Borel $\sigma$-algebra $\calB$ and probability measure $\P$, then $\P$ is regular.
\end{Theorem}
\begin{proof}
Fix $\epsilon >0$ and let $\{x_i\}_{i\geq 1}$ be a dense subset of $\Omega$. Then for any $m\geq 1$, we have
\begin{align*}
\Omega = \bigcup_{i\geq 1} \overline{B}(x_i,1/m),
\end{align*}
where $\overline{B}(x_i,1/m)$ is the closed ball of radius $1/m$ centered at $x_i$. Since $\P(\Omega)=1$, there exists $n(m)$ sufficiently large so that 
\begin{align*}
\P(\Omega\backslash\bigcup_{i=1}^{n(m)} \overline{B}(x_i,\ofrac{m})) \leq \frac{\epsilon}{2^m}.
\end{align*}
Let $K = \displaystyle\bigcap_{m\geq 1}\bigcup_{i=1}^{n(m)} \overline{B}(x_i,1/m)$, which is closed and totally bounded. Since $\Omega$ is complete, $K$ is also complete and hence compact. We then have
\begin{align*}
\P(\Omega\backslash K) 
&= \P(\bigcap_{m\geq 1} \left(\Omega\backslash\bigcup_{i=1}^{n(m)} \overline{B}(x_i,1/m)\right))\\
&\leq \sum_{m\geq 1} \frac{\epsilon}{2^m}\\
&\leq \epsilon.
\end{align*}
From \cref{wk3:lem:closed_approx}, for any $A\in\calA$, there exists a closed $F\subset A$ such that $\P(A\backslash F) \leq \epsilon$. Therefore, 
\begin{align*}
\P(A\backslash(F\cap K)) \leq 2\epsilon,
\end{align*}
and $F\cap K \subset A$ is a compact set. The theorem is now proved. 
\end{proof}


\section{Notes}

\begin{enumerate}[(a)]
	\item $([0,1],\calB([0,1]),\lambda)$ is a probability space, where $\lambda((a,b)) = b-a$ is called the Lebesgue measure. This is the uniform distribution on $[0,1]$.
	\item Let $f: \Real\mapsto\Real_+$ be a function whose set of discontinuities has Lebesgue measure zero and $\int_{-\infty}^\infty f(x) \ud x=1$. Then $(\Real, \calB(\Real), \P)$ where $\P(A) = \int_A f(x) \ud x$, is a probability space. $f$ is an example of a \emph{probability density function} (pdf).
	\item Let $\calX$ be a discrete set. Then $(\calX, 2^\calX, \P)$ where $\P(\{x\}) = p(x)$ with $\sum_{x\in\calX} p(x) =1$, is a probability space. Here, $2^\calX$ denotes the power set of $\calX$, or the collection of all subsets of $\calX$.
	\item There exist non-measureable sets, i.e., one cannot assign a measure to these sets without running into logical consistency issues (see \cref{wk3:example1} or Durrett). This is why the existence of probability spaces is non-trivial as we cannot simply define a measure over the power set $2^\Omega$. 
	\item Exercise: If $A$ is an algebra, then for any $B\in\sigma(A)$, $\exists B_n\in A$ such that 
	\begin{align*}
	\lim_{n\to\infty} \P(B\triangle B_n) = 0,
	\end{align*}
	where $B\triangle B_n = (B\cup B_n)\setminus(B\cap B_n)$ is the symmetric difference of two sets.
\end{enumerate}

%\bibliography{mybib}
\bibliographystyle{alpha}

\end{document}
